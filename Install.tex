%   ファイルおよびフォルダ名(ディレクトリ名)の表記には\verbコマンドを用いる.
%   引数を持たないコマンドは基本的に波括弧で囲う.(e.g. {\hoge})
%   一行あたりの最大文字数は135文字を目安に折り返す.

\addtocounter{page}{-1}

%%%%%%%%%%%%%%%%%%%%%% §1 %%%%%%%%%%%%%%%%%%%%%%
\chapter{インストール}

%%%%%%%%%%%%%%%%%%%%% §1.1 %%%%%%%%%%%%%%%%%%%%%
\section{{\TeX}のインストール}

{\ketcindy}は{\TeX}をベースとした教材作成支援システムとして開発された.
そこで,{\ketcindy}をインストールする前に,Windowsにおける{\TeX}のインストール手順を説明する.
{\TeX}とは,コンピュータ上で出版物を作成できる組版システムのことである.
{\TeX}のフルスペック版は{\TeX}Liveとして纏められたおり,毎年アップデートされ,最新版は{\TeX}Live2023である.

ところで,{\ketcindy}ではフルスペックの{\TeX}を必要としていないため,最低限必要な部分だけ組み込んだ簡易版としてKeT{\TeX}が開発された.
KeT{\TeX}は1.6GB程度とサイズが小さく,日本語で文書を作成する程度なら全く問題ない.
フルスペック版に拘らなければ,ディスク容量を軽減できるため,KeT{\TeX}がお勧めである.

%%%%%%%%%%%%%%%%%%%% §1.1.1 %%%%%%%%%%%%%%%%%%%%
\subsection{{\TeX}Liveの場合}

\begin{enumerate}
    \item \url{https://www.tug.org}にアクセスする
    \item 右ペインにあるSoftwareからTeX Liveを選択する
    \item install on Windowsを選択する
    \item \verb|install-tl-windows.exe|を選択する
    \item exeファイルを管理者として実行\footnote{右クリックメニューから[管理者として実行]を選択}する
    \item インストーラに従いインストールを完了させる
\end{enumerate}

インストール先にCドライブ直下を指定すると{\ketcindy}のインストール時に楽である.
インターネット上から膨大な量のファイルをダウンロードするため,かなりの時間を要するはずである.

%%%%%%%%%%%%%%%%%%%% §1.1.2 %%%%%%%%%%%%%%%%%%%%
\subsection{KeT{\TeX}の場合}

\begin{enumerate}
    \item \url{https://github.com/ketpic/kettex}にアクセスする
    \item 右ペインからReleasesを選択する
    \item \verb|KeTTeX-windows-20230320.zip|を選択する
    \item ダウンロードしたzipファイルをCドライブ直下に移動する
    \item zipファイルを解凍しする
    \item 中にある\verb|kettexinst.cmd|を管理者として実行する
    \item インストーラに従いインストールを完了させる
\end{enumerate}

%%%%%%%%%%%%%%%%%%%%% §1.2 %%%%%%%%%%%%%%%%%%%%%
\section{{\ketcindy}のインストール}

{\ketcindy}は,動的幾何学ソフトCinderellaを用いて図形を描き,{\TeX}の図版ファイルを作るためのシステムである.
gccやMaximaと連携することで,より複雑な処理をすることも可能である.

ここでは,{\ketcindy}本体だけでなくMaximaやMinGW-w64等の周辺ソフトウェアも含めたインストール手順を説明する.

%%%%%%%%%%%%%%%%%%%% §1.2.1 %%%%%%%%%%%%%%%%%%%%
\subsection{Windowsの場合}

ここでは,64bit版Windowsにおける{\ketcindy}のインストール手順を説明する.
特に説明されていない限り,インストール先フォルダは\verb|C:\Program Files|または\verb|C:\Program Files (x86)|を推奨する.

{\ketcindy}を利用する際に必要なソフトウェア\footnote{太字のものは必須}およびURLの一覧を下に示す.

\begin{table}[h]
    \centering
    \caption{ソフトウェア一覧}
    \label{tab:download}
    \begin{tabular}{c||l}
        \textbf{\ketcindy}   & \url{https://github.com/ketpic/ketcindy}\\
        \textbf{R}           & \url{https://cran.r-project.org/}\\
        \textbf{Cinderella}  & \url{https://cinderella.de/tiki-index.php}\\
        Maxima      & \url{https://maxima.sourceforge.io/}\\
        MinGW-w64   & \url{https://www.mingw-w64.org/}\\
        emath       & \url{http://emath.s40.xrea.com/}\\
        SumatraPDF  & \url{https://www.sumatrapdfreader.org/free-pdf-reader}
    \end{tabular}
\end{table}

\subsubsection{{\ketcindy}本体のインストール}
\begin{enumerate}
    \item 表\ref{tab:download}のURLにアクセスする
    \item 右ペインからReleasesを選択する
    \item Source code (zip)を選択する
    \item ダウンロードしたzipファイルをCドライブ直下に移動する
    \item zipファイルを解凍する
    \item 解凍されたフォルダ名から「\verb|-|」(ハイフン)を取り除く
\end{enumerate}

\subsubsection{Rのインストール}
\begin{enumerate}
    \item 表\ref{tab:download}のURLにアクセスする
    \item Download R for Windows→baseと進む
    \item Download R-4.2.3 for Windowsを選択する
    \item ダウンロードしたexeファイルを実行する
    \item インストーラに従いインストールを完了させる
\end{enumerate}

\subsubsection{Cinderellaのインストール}
\begin{enumerate}
    \item 表\ref{tab:download}のURLにアクセスする
    \item 右ペインから ダウンロード を選択する
    \item installer for 64bitを選択する
    \item ダウンロードしたexeファイルを管理者として実行する
    \item インストーラに従いインストールを完了させる
\end{enumerate}

\subsubsection{Maximaのインストール}
\begin{enumerate}
    \item 表\ref{tab:download}のURLにアクセスする
    \item 右ペインからDownloadsを選択する
    \item Installation of Maxima in Windows→5.46.0-Windowsと進む
    \item \verb|maxima-5.46.0-win64.exe|を選択する
    \item ダウンロードしたexeファイルを管理者として実行する
    \item インストーラに従いインストールを完了させる
\end{enumerate}

\subsubsection{MinGW-w64のインストール}
\begin{enumerate}
    \item 表\ref{tab:download}のURLにアクセスする
    \item 左ペインからDownloadsを選択する
    \item 右ペインからMingw-buildsを選択する
    \item GitHubを選択する
    \item \verb|x86_64-12.2.0-release-posix-seh-ucrt-rt_v10-rev2.7z|を選択する
    \item ダウンロードした7zファイルを解凍する
    \item 中にある\verb|bin|フォルダの絶対パスをシステム環境変数として登録する
\end{enumerate}

\subsubsection{emathのインストール}
\begin{enumerate}
    \item 表\ref{tab:download}のURLにアクセスする
    \item 入口→丸ごとパック と進む
    \item \verb|emathf051107c.zip|を選択する
    \item ダウンロードしたzipファイルを解凍する
    \item 中にある\verb|sty.zip|を解凍する
    \item \verb|sty|フォルダを{\ketcindy}フォルダ
          \cprotect\footnote{上の手順に従っている場合,\verb|ketcindyx.x.x|(\verb|x|はバージョン)}内にある\verb|doc\foremathJ|に入れる
\end{enumerate}

\subsubsection{PDFビューアのインストール}
ここではSumatraPDFをインストールする.Adobe Acrobat等のPDFビューアも使用可能だが,
PDFファイルを開いている間にファイルがロックされない\footnote{PDFファイルを開いていても外部でファイルの編集ができる}ため,
特にこだわりがなければSumatraPDFがお薦めである.
\begin{enumerate}
    \item 表\ref{tab:download}のURLにアクセスする
    \item Downloadを選択する
    \item \verb|SumatraPDF-3.4.6-64-install.exe|を選択する
    \item ダウンロードしたexeファイルを実行する
    \item インストール先フォルダを\verb|C:\Program Files\SumatraPDF|に変更する
    \item インストーラに従いインストールを完了させる
\end{enumerate}

%%%%%%%%%%%%%%%%%%%% §1.2.2 %%%%%%%%%%%%%%%%%%%%
\subsection{Macintoshの場合}

