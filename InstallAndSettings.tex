%! suppress = MissingLabel
%   ファイルおよびフォルダ名(ディレクトリ名)の表記には\verbコマンドを用いる.
%   引数を持たないコマンドは基本的に波括弧で囲う.(e.g. {\hoge})
%   一行あたりの最大文字数は135文字を目安に折り返す.

%%%%%%%%%%%%%%%%%%%%%% §2 %%%%%%%%%%%%%%%%%%%%%%
\chapter{{\ketcindy}のインストール}

%%%%%%%%%%%%%%%%%%%%% §2.1 %%%%%%%%%%%%%%%%%%%%%
\section{ダウンロード}

{\ketcindy}は主にGitHub上で開発が行われている.リポジトリのURLを下に示す.

\centerline{\url{https://github.com/ketpic/ketcindy}}

%%%%%%%%%%%%%%%%%%%% §2.1.1 %%%%%%%%%%%%%%%%%%%%
\subsection{Windowsの場合}
\begin{enumerate}
    \item リポジトリにアクセスする
    \item 右ペインからReleasesを選択する
    \item Source code (zip)を選択する
    \item ダウンロードしたzipファイルをCドライブ直下に移動する
    \item zipファイルを解凍する
    \item 解凍されたフォルダ名から「\verb|-|」(ハイフン)を取り除く
\end{enumerate}

%%%%%%%%%%%%%%%%%%%% §2.1.2 %%%%%%%%%%%%%%%%%%%%
\subsection{Macintoshの場合}
\begin{enumerate}
    \item リポジトリにアクセスする
    \item 右ペインからReleasesを選択する
    \item Source code (tar.gz)を選択する
    \item ダウンロードしたtar.gzファイルを解凍する
\end{enumerate}

\newpage

%%%%%%%%%%%%%%%%%%%% §2.1.3 %%%%%%%%%%%%%%%%%%%%
\subsection{Linuxの場合}
\begin{enumerate}
    \item リポジトリにアクセスする
    \item 右ペインからReleasesを選択する
    \item Source code (tar.gz)を選択する
    \item ターミナルから次のコマンドを実行し,任意のディレクトリ(ここでは\verb|~|)に解凍する
    \begin{lstlisting}
  $ tar xvzf ketcindy-4.4.32.tar.gz -C ~
    \end{lstlisting}
\end{enumerate}

%%%%%%%%%%%%%%%%%%%%% §2.2 %%%%%%%%%%%%%%%%%%%%%
\section{インストール}

{\ketcindy}のインストール手順を説明する.

%%%%%%%%%%%%%%%%%%%% §2.2.1 %%%%%%%%%%%%%%%%%%%%
\subsection{Windows・Linuxの場合}
\begin{enumerate}
    \item {\ketcindy}フォルダを開く
    \item \verb|doc|フォルダ内にある\verb|ketcindysettings.cdy|を開く
    \item Langを押して言語を選択する
    \item Texを押して図形の出力に使用する{\LaTeX}エンジンを選択する
    \item Graphicを押して描画に使用するパッケージを選択する
    \item 使用する{\TeX}ディストリビューションをTexlive,Kettex,Otherのいずれかから選択する
    \item Mkinitを押す
    \item Updateを押す
    \item Workを押す
\end{enumerate}

%%%%%%%%%%%%%%%%%%%% §2.2.2 %%%%%%%%%%%%%%%%%%%%
\subsection{Macintoshの場合}
\begin{enumerate}
    \item {\ketcindy}フォルダを開く
    \item \verb|doc|フォルダ内にある\verb|ketcindysettings.cdy|を開く
    \item Langを押して言語を選択する
    \item Texを押して図形の出力に使用する{\LaTeX}エンジンを選択する
    \item Graphicを押して描画に使用するパッケージを選択する
    \item Mackcを押してshに切り替える
    \item 使用する{\TeX}ディストリビューションをTexlive,Kettex,Otherのいずれかから選択する
    \item Mkinitを押す
    \item Updateを押す
    \item Workを押す
\end{enumerate}

\newpage

%%%%%%%%%%%%%%%%%%%%% §2.3 %%%%%%%%%%%%%%%%%%%%%
\section{オプション・emathとの連携}
数式で用いる記号の形が,日本で一般的に用いられるものと微妙に異なっている場合がある.
日本の数学教科書の書式を利用したい場合は,emathを用いるとよい.

形状が違う記号の例を表\ref{tab:math-symbol}に示す.
\begin{table}[h]
    \renewcommand{\arraystretch}{1.5}
    \centering
    \caption{数学記号の形状比較}
    \label{tab:math-symbol}
    \begin{tabular}{c||lc|lc}
                 & \multicolumn{2}{c|}{標準}                      & \multicolumn{2}{c}{emath}                          \\
        \hline
        等号否定 & \verb|\ne|              & $\neq$               & \verb|\neqq|              & $\neqq$                \\
        分数     & \verb|\dfrac{1}{2}|     & $\dfrac{1}{2}$       & \verb|\bunsuu{1}{2}|      & $\bunsuu{1}{2}$        \\
        ベクトル & \verb|\vec{a}, \vec{b}| & $\vec{a}$, $\vec{b}$ & \verb|\beku{a}, \beku{b}| & $\beku{a}$, $\beku{b}$
    \end{tabular}
\end{table}

emathのインストール手順を下に示す.
\begin{enumerate}
    \item \url{http://emath.s40.xrea.com/}にアクセスする
    \item こちら→入口→丸ごとパック と進み,\verb|emathf051107c.zip|を選択する
    \item ダウンロードした\verb|emathf051107c.zip|を解凍する
    \item 中にある\verb|sty.zip|を解凍する
    \item \verb|sty|フォルダを{\ketcindy}のフォルダ内にある\verb|doc\foremathJ|に入れる
    \item \verb|copymath.cdy|を開く
    \item 使用する{\TeX}ディストリビューションを選択する
    \item CopyEmathをクリックする
\end{enumerate}

