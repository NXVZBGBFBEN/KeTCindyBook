%   ファイルおよびフォルダ名(ディレクトリ名)の表記には\verbコマンドを用いる.
%   引数を持たないコマンドは基本的に波括弧で囲う.(e.g. {\hoge})
%   一行あたりの最大文字数は135文字を目安に折り返す.

%%%%%%%%%%%%%%%%%%%%%% §2 %%%%%%%%%%%%%%%%%%%%%%
\chapter{{\ketcindy}のインストール}

%%%%%%%%%%%%%%%%%%%%% §2.1 %%%%%%%%%%%%%%%%%%%%%
\section{ダウンロード}

{\ketcindy}は主にGitHub上で開発が行われている.リポジトリのURLを下に示す.

\centerline{\url{https://github.com/ketpic/ketcindy}}

%%%%%%%%%%%%%%%%%%%% §2.1.1 %%%%%%%%%%%%%%%%%%%%
\subsection{Windowsの場合}
\begin{enumerate}
    \item リポジトリにアクセスする
    \item 右ペインからReleasesを選択する
    \item Source code (zip)を選択する
    \item ダウンロードしたzipファイルをCドライブ直下に移動する
    \item zipファイルを解凍する
    \item 解凍されたフォルダ名から「\verb|-|」(ハイフン)を取り除く
\end{enumerate}

%%%%%%%%%%%%%%%%%%%% §2.1.2 %%%%%%%%%%%%%%%%%%%%
\subsection{Macintoshの場合}
\begin{enumerate}
    \item リポジトリにアクセスする
    \item 右ペインからReleasesを選択する
    \item Source code (tar.gz)を選択する
    \item ダウンロードしたtar.gzファイルを解凍する
\end{enumerate}

\newpage

%%%%%%%%%%%%%%%%%%%%% §2.2 %%%%%%%%%%%%%%%%%%%%%
\section{インストール}

{\ketcindy}のインストール手順を説明する.

%%%%%%%%%%%%%%%%%%%% §2.2.1 %%%%%%%%%%%%%%%%%%%%
\subsection{emathとの連携}
emathとの連携を利用する場合は,下記の手順に従って適切な位置にファイルを配置する必要がある.
\begin{enumerate}
    \item 事前にダウンロードした\verb|emathf051107c.zip|を解凍する
    \item 中にある\verb|sty.zip|を解凍する
    \item \verb|sty|フォルダを{\ketcindy}のフォルダ内にある\verb|doc\foremathJ|に入れる
\end{enumerate}

%%%%%%%%%%%%%%%%%%%% §2.2.2 %%%%%%%%%%%%%%%%%%%%
\subsection{Windowsの場合}

%%%%%%%%%%%%%%%%%%%% §2.2.3 %%%%%%%%%%%%%%%%%%%%
\subsection{Macintoshの場合}



\begin{comment}
\section{基本設定}
{\ketcindy}を利用するための初期設定は,同梱の設定用ファイルを通して行う.

\cprotect\subsection{\verb|ketcindysettings.cdy|による設定}
\begin{enumerate}
    \item {\ketcindy}フォルダを開く
    \item \verb|doc|フォルダ内にある\verb|ketcindysettings.cdy|を開く
\end{enumerate}
\end{comment}