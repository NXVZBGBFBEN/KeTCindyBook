%   ファイルおよびフォルダ名(ディレクトリ名)の表記には\verbコマンドを用いる.
%   引数を持たないコマンドは基本的に波括弧で囲う.(e.g. {\hoge})
%   一行あたりの最大文字数は135文字を目安に折り返す.

%%%%%%%%%%%%%%%%%%%%%% §3 %%%%%%%%%%%%%%%%%%%%%%
\chapter{使い方}

ここでは,{\ketcindy}に同梱されているサンプルを用いて,使い方の例をいくつか紹介する.

サンプルの場所はホームディレクトリ\footnote{Windowsの場合\verb|C:\Users\ユーザ名|,Mac/Linuxの場合\verb|/home/ユーザ名|}内の
\verb|ketcindyYYYYMmmDD/samples|\footnote{例: \verb|ketcindy2022Dec25/samples|}である.
(以降,このフォルダを「サンプルディレクトリ」と呼ぶことにする.)

%%%%%%%%%%%%%%%%%%%%% §3.1 %%%%%%%%%%%%%%%%%%%%%
\section{基本描画}
まずは,{\ketcindy}がどのようなものなのかを体験してもらうため,基本的な2D図形の描画について紹介する.

サンプルディレクトリ内の\verb|s01figure/|に移動してほしい.

%%%%%%%%%%%%%%%%%%%% §3.1.1 %%%%%%%%%%%%%%%%%%%%
\subsection{三角形の外接円}
\verb|s0101figure.cdy|には,三角形の外接円が収録されている.

Cinderellaでこのファイルを開き,上部にあるFigureボタンを押すと,作成した図形がへ\verb|.tex|形式で出力される.
出力される場所は\verb|fig/|である.

\begin{layer}{180}{0}
    \putnotese{0}{5}{\includegraphics[width=77.5mm]{./out/fig/s0101figure-1}}
    \putnotesw{160}{5}{\includegraphics[width=77.5mm]{./out/fig/s0101figure-2}}
\end{layer}
